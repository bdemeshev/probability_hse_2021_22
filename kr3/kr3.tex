% arara: xelatex

\documentclass[12pt]{article} % размер шрифта

\usepackage{etex} % extend
\usepackage{tikz} % картинки в tikz
\usepackage{microtype} % свешивание пунктуации

\usepackage{diagbox}
\usepackage{slashbox}
\usepackage{tabularx}
\usepackage{comment}

\usepackage{tikzlings}
\usepackage{tikzducks}

\usepackage{array} % для столбцов фиксированной ширины
\usepackage{verbatim} % для вставки комментариев

\usepackage{indentfirst} % отступ в первом параграфе

\usepackage{sectsty} % для центрирования названий частей

\allsectionsfont{\centering} % приказываем центрировать все sections

\usepackage{amsmath,  amsfonts} % куча стандартных математических плюшек

\usepackage[top=1.5cm,  left=1.5cm,  right=1.5cm,  bottom=1.5cm]{geometry} % размер текста на странице

\usepackage{lastpage} % чтобы узнать номер последней страницы

\usepackage{enumitem} % дополнительные плюшки для списков
%  например \begin{enumerate}[resume] позволяет продолжить нумерацию в новом списке
\usepackage{caption} % подписи к картинкам без плавающего окружения figure

\usepackage{comment} % длинные комментарии

\usepackage{fancyhdr} % весёлые колонтитулы
\pagestyle{fancy}
\lhead{Теория вероятностей и статистика,  НИУ-ВШЭ}
\chead{}
\rhead{21 марта 2022 года, контрольная работа 3}
\lfoot{вариант $\kappa$}
\cfoot{}
\rfoot{\thepage}
\renewcommand{\headrulewidth}{0.4pt}
\renewcommand{\footrulewidth}{0.4pt}

\usepackage{todonotes} % для вставки в документ заметок о том,  что осталось сделать
% \todo{Здесь надо коэффициенты исправить}
% \missingfigure{Здесь будет картина Последний день Помпеи}
% команда \listoftodos — печатает все поставленные \todo'шки

\usepackage{booktabs} % красивые таблицы
% заповеди из документации:
% 1. Не используйте вертикальные линии
% 2. Не используйте двойные линии
% 3. Единицы измерения помещайте в шапку таблицы
% 4. Не сокращайте .1 вместо 0.1
% 5. Повторяющееся значение повторяйте,  а не говорите "то же"

\usepackage{fontspec} % поддержка разных шрифтов
\usepackage{polyglossia} % поддержка разных языков

\setmainlanguage{russian}
\setotherlanguages{english}

\setmainfont{Linux Libertine O} % выбираем шрифт

% можно также попробовать Helvetica,  Arial,  Cambria и т.Д.

% чтобы использовать шрифт Linux Libertine на личном компе, 
% его надо предварительно скачать по ссылке
% http://www.linuxlibertine.org/index.php?id=91&L=1

\newfontfamily{\cyrillicfonttt}{Linux Libertine O}
% пояснение зачем нужно шаманство с \newfontfamily
% http://tex.stackexchange.com/questions/91507/

\AddEnumerateCounter{\asbuk}{\russian@alph}{щ} % для списков с русскими буквами
\setlist[enumerate,  2]{label=\asbuk*), ref=\asbuk*} % списки уровня 2 будут буквами а) б) \ldots 

%% эконометрические и вероятностные сокращения
\DeclareMathOperator{\Cov}{Cov}
\DeclareMathOperator{\Corr}{Corr}
\DeclareMathOperator{\Var}{Var}
\DeclareMathOperator{\E}{\mathbb{E}}
\DeclareMathOperator{\D}{D}
\newcommand \hb{\hat{\beta}}
\newcommand \hs{\hat{\sigma}}
\newcommand \htheta{\hat{\theta}}
\newcommand \s{\sigma}
\newcommand \hy{\hat{y}}
\newcommand \hY{\hat{Y}}
\newcommand \e{\varepsilon}
\newcommand \he{\hat{\e}}
\newcommand \z{z}
\newcommand \hVar{\widehat{\Var}}
\newcommand \hCorr{\widehat{\Corr}}
\newcommand \hCov{\widehat{\Cov}}
\newcommand \cN{\mathcal{N}}
\newcommand{\R}{\mathbb{R}}

\let\P\relax
\DeclareMathOperator{\P}{\mathbb{P}}

%\fbox{
%  \begin{minipage}{24em}
%    Фамилия,  имя и номер группы (печатными буквами):\vspace*{3ex}\par
%    \noindent\dotfill\vspace{2mm}
%  \end{minipage}
%  \begin{tabular}{@{}l p{0.8cm} p{0.8cm} p{0.8cm} p{0.8cm} p{0.8cm}@{}}
% %\toprule
% Задача & 1 & 2 & 3 & 4 & 5\\ 
% \midrule
% Балл  &  &  & & & \\
% \midrule
% %\bottomrule
% \end{tabular}
% }    


\begin{document}


\lfoot{вариант $\kappa$}
\setcounter{page}{1}

\begin{center}
    \textbf{Основная часть}
\end{center}


\begin{enumerate}


\item \textbf{(5 баллов)} Пусть $X_1, \dots, X_n$ --- случайная выборка из распределения с плотностью

\begin{equation*}
f(x, \theta) = 
 \begin{cases}
   \frac{3x^2}{\theta^3} &\text{при $x \in [0,\theta]$,}\\
   0 &\text{иначе,}
 \end{cases}
\end{equation*}

где $\theta > 0$. Используя центральный момент второго порядка, при помощи метода моментов найдите оценку для неизвестного параметра $\theta$.

\item \textbf{(10 баллов)} Случайные величины $X, Y, Z$ независимы и нормально распределены, $N(0, 4)$.

\[
\gamma_1 = X^2 + Y^2 + Z^2,
\]

\[
\gamma_2 = \frac{X}{\sqrt{Y^2 + Z^2}}.
\]

Найдите $\mathbb{E}(\gamma_1)$, $\mathbb{E}(\gamma_2)$, Var($\gamma_1$),  Var($\gamma_2$), $\mathbb{P}\{\gamma_1 > 2.34\}$, $\mathbb{P}\{\gamma_2 < 0.75\}$.

\item[3.1.] \textbf{(5 баллов)} Случайный опрос показал, что 87 студентов из 100 с симпатией относятся к Джокеру. Постройте доверительный интервал для истинной доли симпатизирующих Джокеру. Можно ли утверждать, что 90\% студентов симпатизируют Джокеру? 

\item[3.2.] \textbf{(5 баллов)} С помощью ЦПТ и теоремы Слуцкого обоснуйте выбор распределения для построения доверительного интервала в предыдущем пункте.

\item[4.] \textbf{(5 баллов)} Студенты Петя, Вася и Маша пообедали в <<Груше>> и в столовой. В таблице приведены стоимости обедов. Считая, что расходы на обед хорошо описываются нормальным распределением, постройте доверительный интервал для разницы математических ожиданий стоимостей обеда в <<Груше>> и столовой.

\begin{tabularx}{0.8\textwidth} { 
  | >{\centering\arraybackslash}X 
  | >{\centering\arraybackslash}X 
  | >{\centering\arraybackslash}X | }
 \hline
      & <<Груша>> & Столовая \\
 \hline
 Петя  & 380  & 350 \\
\hline
 Вася  & 600  & 450  \\
\hline
Маша  & 250  & 350  \\
\hline
\end{tabularx}

\item[5.] \textbf{(30 баллов)} По выборке $X=(X_1, \dots, X_n)$ из нормального распределения, $N(0, \theta)$

\begin{enumerate}

    \item \textbf{(7 баллов)} методом максимального правдоподобия оцените параметр $\theta$,
    
    \item \textbf{(8 баллов)} вычислите математическое ожидание и дисперсию найденной оценки,
    
    \item \textbf{(2 балла)} проверьте несмещенность оценки,
    
    \item \textbf{(1 балл)} проверьте, будет ли оценка асимптотически несмещена,
    
    \item \textbf{(2 балла)} проверьте состоятельность оценки,
    
    \item \textbf{(2 балла)} вычислите информацию Фишера, содержащуюся в выборке,
    
    \item \textbf{(2 балла)} проверьте, является ли найденная оценка эффективной в соответствующем классе,
    
    \item \textbf{(2 балла)} найдите оценку максимального правдоподобия для стандартного отклонения $X_1$,
    
    \item \textbf{(4 балла)} найдите асимптотическую дисперсию оценки из предыдущего пункта.
    
\end{enumerate}


\end{enumerate}


\end{document}