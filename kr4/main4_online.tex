  % arara: xelatex

\documentclass[12pt]{article} % размер шрифта

\usepackage{etex} % extend
\usepackage{tikz} % картинки в tikz
\usepackage{microtype} % свешивание пунктуации

\usepackage{diagbox}
\usepackage{slashbox}
\usepackage{tabularx}
\usepackage{comment}

\usepackage{tikzlings}
\usepackage{tikzducks}

\usepackage{array} % для столбцов фиксированной ширины
\usepackage{verbatim} % для вставки комментариев

\usepackage{indentfirst} % отступ в первом параграфе

\usepackage{sectsty} % для центрирования названий частей

\allsectionsfont{\centering} % приказываем центрировать все sections

\usepackage{amsmath,  amsfonts} % куча стандартных математических плюшек

\usepackage[top=1.9cm,  left=1.5cm,  right=1.5cm,  bottom=1.9cm]{geometry} % размер текста на странице

\usepackage{lastpage} % чтобы узнать номер последней страницы

\usepackage{enumitem} % дополнительные плюшки для списков
%  например \begin{enumerate}[resume] позволяет продолжить нумерацию в новом списке
\usepackage{caption} % подписи к картинкам без плавающего окружения figure

\usepackage{comment} % длинные комментарии

\usepackage{fancyhdr} % весёлые колонтитулы
\pagestyle{fancy}
\lhead{Теория вероятностей и статистика,  НИУ-ВШЭ}
\chead{}
\rhead{23 мая 2022 года, контрольная работа 4}
\lfoot{Вариант $\alpha$}
\cfoot{ПАНИКОВАТЬ СТРОГО ЗАПРЕЩАЕТСЯ}
\rfoot{\thepage}
\renewcommand{\headrulewidth}{0.4pt}
\renewcommand{\footrulewidth}{0.4pt}

\usepackage{todonotes} % для вставки в документ заметок о том,  что осталось сделать
% \todo{Здесь надо коэффициенты исправить}
% \missingfigure{Здесь будет картина Последний день Помпеи}
% команда \listoftodos — печатает все поставленные \todo'шки

\usepackage{booktabs} % красивые таблицы
% заповеди из документации:
% 1. Не используйте вертикальные линии
% 2. Не используйте двойные линии
% 3. Единицы измерения помещайте в шапку таблицы
% 4. Не сокращайте .1 вместо 0.1
% 5. Повторяющееся значение повторяйте,  а не говорите "то же"

\usepackage{fontspec} % поддержка разных шрифтов
\usepackage{polyglossia} % поддержка разных языков

\setmainlanguage{russian}
\setotherlanguages{english}

\setmainfont{Linux Libertine O} % выбираем шрифт

% можно также попробовать Helvetica,  Arial,  Cambria и т.Д.

% чтобы использовать шрифт Linux Libertine на личном компе, 
% его надо предварительно скачать по ссылке
% http://www.linuxlibertine.org/index.php?id=91&L=1

\newfontfamily{\cyrillicfonttt}{Linux Libertine O}
% пояснение зачем нужно шаманство с \newfontfamily
% http://tex.stackexchange.com/questions/91507/

\AddEnumerateCounter{\asbuk}{\russian@alph}{щ} % для списков с русскими буквами
\setlist[enumerate,  2]{label=\asbuk*), ref=\asbuk*} % списки уровня 2 будут буквами а) б) \ldots 

%% эконометрические и вероятностные сокращения
\DeclareMathOperator{\Cov}{Cov}
\DeclareMathOperator{\Corr}{Corr}
\DeclareMathOperator{\Var}{Var}
\DeclareMathOperator{\E}{\mathbb{E}}
\DeclareMathOperator{\D}{D}
\newcommand \hb{\hat{\beta}}
\newcommand \hs{\hat{\sigma}}
\newcommand \htheta{\hat{\theta}}
\newcommand \s{\sigma}
\newcommand \hy{\hat{y}}
\newcommand \hY{\hat{Y}}
\newcommand \e{\varepsilon}
\newcommand \he{\hat{\e}}
\newcommand \z{z}
\newcommand \hVar{\widehat{\Var}}
\newcommand \hCorr{\widehat{\Corr}}
\newcommand \hCov{\widehat{\Cov}}
\newcommand \cN{\mathcal{N}}
\newcommand{\R}{\mathbb{R}}

\let\P\relax
\DeclareMathOperator{\P}{\mathbb{P}}

%\fbox{
%  \begin{minipage}{24em}
%    Фамилия,  имя и номер группы (печатными буквами):\vspace*{3ex}\par
%    \noindent\dotfill\vspace{2mm}
%  \end{minipage}
%  \begin{tabular}{@{}l p{0.8cm} p{0.8cm} p{0.8cm} p{0.8cm} p{0.8cm}@{}}
% %\toprule
% Задача & 1 & 2 & 3 & 4 & 5\\ 
% \midrule
% Балл  &  &  & & & \\
% \midrule
% %\bottomrule
% \end{tabular}
% }    


\begin{document}
\begin{enumerate}
\item Разработанный в порядке импортозамещения алгоритм прогнозирования погоды Глафира предсказывает шесть дней подряд, будет ли дождь на следующий день. Глафира верно угадывает погоду с неизвестной вероятностью $p$ каждый раз независимо от других.

Величины $X_1$, $X_2$, \ldots, $X_6$ равны 1, если Глафира угадала, и 0, если ошиблась. 

Для вступления в Российский Клуб Гадалок алгоритм Глафира проходит тест Тьюринга:

Если $\sum_{i=1}^6 X_i \leq 4$, то гипотеза $H_0$ о $p=0.5$ не отвергается. 

Если $\sum_{i=1}^6 X_i \geq 5$, то гипотеза $H_0$ отвергается в пользу альтернативной гипотезы о высоком мастерстве и Глафиру принимают в клуб.

\begin{enumerate}
    \item (5 баллов) Найдите вероятность ошибки первого рода.
    \item (5 баллов) Найдите вероятность ошибки второго рода,
    если альтернативная гипотеза состоит в том, что $p=0.8$.
    \item (5 баллов) Найдите функцию мощности критерия в зависимости от значения $p$ в альтернативной гипотезе. 
\end{enumerate}

\item Обновленный алгоритм Авдотья предсказывает 100 дней подряд, будет ли дождь на следующий день. Она верно угадывает погоду с неизвестной вероятностью $p \in (0.5; 1)$ каждый раз независимо от других. 
Авдотье удалось угадать погоду в 60 случаях. 

% Снова рассмотрим величины $X_i$, равные 1, если Глафира угадала на $i$-й день и $0$ иначе. 

\begin{enumerate}
\item (10 баллов) Постройте асимптотический 95\%-й доверительный интервал для параметра $p$, а затем преобразуйте его в интервал для параметра $a= p/(1-p)$ без использования дельта-метода.
\item (10 баллов) С помощью дельта-метода постройте приближенный 95\%-ый доверительный интервал для дисперсии числа угадываний при 10 попытках.
\end{enumerate}



\item Удовлетворенность кота Матроскина рыбалкой, случайная величина $X$, зависит от неизвестного Шарику параметра $\theta$, числа пойманных рыб. 
Плотность величины $X$ имеет вид 
\[
f(x \mid \theta) = \begin{cases}
\theta x^{\theta-1}, \text{ при } x\in [0;1], \\
0, \text{ иначе.}
\end{cases}
\]
Шарик хочет протестировать гипотезу $H_0$: $\theta = 2$ против альтернативной $H_1$: $\theta=1$, зная лишь значение $X$ после одной рыбалки. 
\begin{enumerate}
\item (10 баллов) С помощью леммы Неймана–Пирсона найдите наиболее мощный критерий, имеющий уровень значимости $\alpha=0.05$.
\item (5 баллов) Рассчитайте мощность найденного вами критерия.
\end{enumerate}

\newpage

\item (10 баллов) Три независимые случайные выборки из трёх наблюдений каждая, $(X_1, X_2, X_3)$, $(Y_1, Y_2, Y_3)$ и $(Z_1, Z_2, Z_3)$,
имеют нормальные распределения с разными ожиданиями и дисперсиями равными $\sigma^2_x=1$, $\sigma^2_y=2$, $\sigma^2_z=3$. 

Постройте 95\%-й доверительный интервал для суммы ожиданий $\mu_x + \mu_y + \mu_z$, если $\bar x= 0.5$, $\bar y= 1.0$ и $\bar z= 1.5$.


\item Ассистенты Петя и Вася проверяют две домашки.
Каждую домашку пишет ровно 45 студентов.

Ассистенты случайно делят работы между собой, так что Пете всегда достаётся 20 работ, а Васе — 25 работ.

Итоги проверок представлены в таблице:

\begin{tabular}{cccc}
\toprule
Домашка & Ассистент & Выборочное среднее & Выборочная дисперсия \\
\midrule 
1 & Вася & 5.6 & $1.0^2$ \\
1 & Петя & 6.3 & $1.2^2$ \\
2 & Вася & 5.9 & $0.9^2$ \\
2 & Петя & 6.5 & $1.1^2$ \\
\bottomrule
\end{tabular}

Оценки, выставляемые Петей и Васей, хорошо описываются нормальным распределением.

\begin{enumerate}
    \item (4 балла) Используя данные по первой домашке, проверьте гипотезу о том, что разброс оценок у Пети и Васи одинаковый против гипотезы о разном разбросе.
    \item (4 балла) Предположим дополнительно, что разбросы оценок Пети и Васи равны. Используя данные по второй домашке, проверьте гипотезу об одинаковой строгости Васи и Пети против гипотезы о большей строгости Васи.
    \item (2 балла) Аккуратно выпишите предположения, использованные при проведении каждого теста.  
\end{enumerate}

%6.1 проверьте гипотезу, что разброс в оценках, выставляемых Васей и Петей, одинаков;
%6.2 проверьте гипотезу о том, что Вася строже Пети.
%6.3 Можно ли было корректно проверить гипотезу предыдущего пункта, если бы были известны только результаты проверки первой домашки?

Для проверки гипотез используйте уровень значимости $0.1$. 

 
\end{enumerate}

\end{document}

\newpage
\lfoot{Вариант $\omega$}
\setcounter{page}{1}


\begin{enumerate}
\item Разработанный в порядке импортозамещения алгоритм прогнозирования погоды Глафира предсказывает семь дней подряд, будет ли дождь на следующий день. Глафира верно угадывает погоду с неизвестной вероятностью $p$ каждый раз независимо от других.

Величины $X_1$, $X_2$, \ldots, $X_7$ равны 1, если Глафира угадала, и 0, если ошиблась. 

Для вступления в Российский Клуб Гадалок алгоритм Глафира проходит тест Тьюринга:

Если $\sum_{i=1}^7 X_i \leq 5$, то гипотеза $H_0$ о $p=0.5$ не отвергается. 

Если $\sum_{i=1}^7 X_i \geq 6$, то гипотеза $H_0$ отвергается в пользу альтернативной гипотезы о высоком мастерстве и Глафиру принимают в клуб.

\begin{enumerate}
    \item (5 баллов) Найдите вероятность ошибки первого рода.
    \item (5 баллов) Найдите вероятность ошибки второго рода,
    если альтернативная гипотеза состоит в том, что $p=0.8$.
    \item (5 баллов) Найдите функцию мощности критерия в зависимости от значения $p$ в альтернативной гипотезе. 
\end{enumerate}

\item Обновленный алгоритм Авдотья предсказывает 100 дней подряд, будет ли дождь на следующий день. Она верно угадывает погоду с неизвестной вероятностью $p \in (0.5; 1)$ каждый раз независимо от других. 
Авдотье удалось угадать погоду в 70 случаях. 

% Снова рассмотрим величины $X_i$, равные 1, если Глафира угадала на $i$-й день и $0$ иначе. 

\begin{enumerate}
\item (10 баллов) Постройте асимптотический 95\%-й доверительный интервал для параметра $p$, а затем преобразуйте его в интервал для параметра $a= p/(1-p)$ без использования дельта-метода.
\item (10 баллов) С помощью дельта-метода постройте приближенный 95\%-ый доверительный интервал для дисперсии числа угадываний при 20 попытках.
\end{enumerate}



\item Удовлетворенность кота Матроскина рыбалкой, случайная величина $X$, зависит от неизвестного Шарику параметра $\theta$, числа пойманных рыб. 
Плотность величины $X$ имеет вид 
\[
f(x \mid \theta) = \begin{cases}
\theta x^{\theta-1}, \text{ при } x\in [0;1], \\
0, \text{ иначе.}
\end{cases}
\]
Шарик хочет протестировать гипотезу $H_0$: $\theta = 3$ против альтернативной $H_1$: $\theta=1$, зная лишь значение $X$ после одной рыбалки. 
\begin{enumerate}
\item (10 баллов) С помощью леммы Неймана–Пирсона найдите наиболее мощный критерий, имеющий уровень значимости $\alpha=0.05$.
\item (5 баллов) Рассчитайте мощность найденного вами критерия.
\end{enumerate}

\newpage

\item (10 баллов) Три независимые случайные выборки из трёх наблюдений каждая, $(X_1, X_2, X_3)$, $(Y_1, Y_2, Y_3)$ и $(Z_1, Z_2, Z_3)$,
имеют нормальные распределения с разными ожиданиями и дисперсиями равными $\sigma^2_x=1$, $\sigma^2_y=2$, $\sigma^2_z=3$. 

Постройте 95\%-й доверительный интервал для суммы ожиданий $\mu_x + \mu_y + \mu_z$, если $\bar x= 1.5$, $\bar y= 2.0$ и $\bar z= 2.5$.


\item Ассистенты Петя и Вася проверяют две домашки.
Каждую домашку пишет ровно 45 студентов.

Ассистенты случайно делят работы между собой, так что Пете всегда достаётся 25 работ, а Васе — 20 работ.

Итоги проверок представлены в таблице:

\begin{tabular}{cccc}
\toprule
Домашка & Ассистент & Выборочное среднее & Выборочная дисперсия \\
\midrule 
1 & Вася & 6.6 & $1.0^2$ \\
1 & Петя & 7.3 & $1.3^2$ \\
2 & Вася & 6.9 & $0.9^2$ \\
2 & Петя & 7.5 & $1.2^2$ \\
\bottomrule
\end{tabular}

Оценки, выставляемые Петей и Васей, хорошо описываются нормальным распределением.

\begin{enumerate}
    \item (4 балла) Используя данные по первой домашке, проверьте гипотезу о том, что разброс оценок у Пети и Васи одинаковый против гипотезы о разном разбросе.
    \item (4 балла) Предположим дополнительно, что разбросы оценок Пети и Васи равны. Используя данные по второй домашке, проверьте гипотезу об одинаковой строгости Васи и Пети против гипотезы о большей строгости Васи.
    \item (2 балла) Аккуратно выпишите предположения, использованные при проведении каждого теста.  
\end{enumerate}

%6.1 проверьте гипотезу, что разброс в оценках, выставляемых Васей и Петей, одинаков;
%6.2 проверьте гипотезу о том, что Вася строже Пети.
%6.3 Можно ли было корректно проверить гипотезу предыдущего пункта, если бы были известны только результаты проверки первой домашки?

Для проверки гипотез используйте уровень значимости $0.1$. 

 
\end{enumerate}



\end{document}
